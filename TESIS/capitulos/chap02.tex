\chapter{Fundamentos teóricos}

Este capítulo incluye la revisión de la literatura, de los enfoques, teorías o conceptos pertinentes en que se fundamenta la investigación. Se basa fundamentalmente en la exposición de otros trabajos sobre el tema estudiado.

En diferentes tradiciones académicas este capítulo recibe distintas denominaciones: Marco teórico, Estado de la cuestión, Estado del arte. El objetivo de este capítulo es guiar al lector en la interpretación de trabajos que se han ocupado previamente de la cuestión central de la tesis u ofrecen herramientas analíticas o interpretativas. 

Algunas disciplinas incluyen aquí objetivos, hipótesis y justificación de la metodología, en tanto que otras exigen capítulos independientes para estos contenidos. Asimismo, algunos trabajos requieren un capítulo propio para los antecedentes de la investigación.



\section{XXXX}

Es usual que en \textit{Fundamentos teóricos} o en otras partes de la tesis el autor incluya vocabulario específico de la disciplina en un glosario.

Un glosario incluye una lista de términos y su explicación sucinta. El objetivo de este apartado es permitirle a un lector especializado en el área, aunque no necesariamente en la temática, comprender con mayor facilidad ciertos términos. 
Se organiza en forma alfabética y en el cuerpo de la obra se lo puede señalar con \textsc{versalita} la primera vez que se menciona, si este tipo de letra no fue utilizado con otro fin.

Se mencionan a modo de ejemplo tres posibles palabras a definir:
%
\begin{itemize}
\item \gls{adjetivo}
\item \gls{adjetivo_re}
\item \gls{adjetivo_cu}
\end{itemize}
