% ===== INICIO DEL PREÁMBULO =====

\documentclass[msc,oneside,a4paper]{udelar} % Poner msc para Maestría, dsc para Doctorado.

\usepackage[
acronyms, 											 %utiliza el glosario de acronimos.
nohypertypes={acronym,notacion,simbolos,glosario},   %quita los links en el texto al glosario.
%nonumberlist,                                       %quita los links en los glosarios al texto.
nogroupskip,                                         %quita los espacios entre diferentes grupos dentro de un glosario.
nopostdot, 											 %quita el punto final en los acrónimos         .                           
]{glossaries}

\hypersetup{ colorlinks = true } % false para imprimir, true para ver.

% Ver los documentos de estilos bibliográficos para editar estas siguientes 2 líneas.
\usepackage{natbib} % Para algunos estilos bibliográficos
\bibliographystyle{estilos_bibliograficos/natbib/apalike}



\loadglossary

% A su vez, la clase udelar.cls ya tiene los siguientes paquetes cargados automáticamente, que podrían ser de interés saber para el usuario:
% 
% {color},{hyphenat},{appendix},{lastpage},{babel},{inputenc},{amsmath,amssymb},{ifthen},{graphicx},{caption}
% {setspace},{tabularx},{eqparbox},{ltxcmds},{titletoc},{xcolor},{lineno},{xwatermark}

% Si se quieren agregar más paquetes, se recomienda colocarlos a partir de esta linea y antes de \begin{document}.



% ===== FIN DEL PREÁMBULO =====

% ===== INICIO DEL DOCUMENTO =====

\begin{document}



  \title{Título de la tesis}
  \subtitle{Subtítulo de la tesis}
  \institutelogo{3} % Carga cantidad de logos seleccionados, con máximo de 3 logos.
  \author{Nombres del autor}{Apellidos}
  
  \director{Prof.}{Nombre del Director de Tesis}{Apellido}{D.Sc.}
  \codirector{Prof.}{Nombre del 1er Codirector}{Apellido}{D.Sc.}  % Comentar estas líneas si no son necesarias.
  \codirector{Prof.}{Nombre del 2do Codirector}{Apellido}{D.Sc.}
  \codirector{Prof.}{Nombre del 3er Codirector}{Apellido}{D.Sc.}
  \directoracademico{Prof.}{Nombre del Director Académico de Tesis}{Apellido}{D.Sc.}

  \examiner{Prof.}{Nombre del 1er Examinador}{Apellido}{D.Sc.}
  \examiner{Prof.}{Nombre del 2do Examinador}{Apellido}{Ph.D.}
  \examiner{Prof.}{Nombre del 3er Examinador}{Apellido}{D.Sc.}
  \examiner{Prof.}{Nombre del 4to Examinador}{Apellido}{Ph.D.}
  \examiner{Prof.}{Nombre del 5to Examinador}{Apellido}{Ph.D.}
  
  \graduatename{Ingeniería Estructural}
  \institute{Facultad de Ingeniería}{FIng}  % La primer institución es la principal.
  \institute{Facultad de Medicina}{FMed}  % Agregar copias de esta línea para agregar instituciones.
  \graduatelocation{Montevideo}{Uruguay}
  
  
  \date{11}{10}{2015} % Fecha del documento.

  \keyword{1ra palabra clave}
  \keyword{2da palabra clave}
  \keyword{3ra palabra clave}
  \keyword{4ta palabra clave}
  \keyword{5ta~palabra clave}
  
  \foreignkeyword{1st keyword}
  \foreignkeyword{2nd keyword}
  \foreignkeyword{3rd keyword}
  \foreignkeyword{4th keyword}
  \foreignkeyword{5th keyword}
        
  \maketitle  % Comando que genera el título de la tesis.
  
  \frontmatter  % Comando que genera la portadilla, el catalogo y el tribunal de evaluación. NO COMENTAR
  
  \dedication{(Dedicatoria) A alguien cuyo valor es digno de ella.}


  \chapter*{Agradecimentos}

Quisiera agredecer a...

  \epigrafe{{\normalfont{(Epígrafe:)}} Frase que alude al tema de trabajo.}{Autor}


  \begin{abstract}

En esta tesis se presenta...

\end{abstract}


  \include{resumen/abstract}
  
  
%  \listoffigures	         % Lista de figuras
%  \listoftables	         % Lista de tablas
  
  \listadesimbolos 		     % Lista de símbolos
  %\listadenotaciones 	     % Lista de notaciones
  \listadesiglas 		     % Lista de siglas

  \tableofcontents           % Tabla de contenidos

  \mainmatter % Comando que genera las listas y capítulos. NO COMENTAR
  
  
  \chapter{Introducción}

Este material busca ser un apoyo a quienes escriben sus tesis en los distintos servicios en las diversas disciplinas que se cultivan en la \gls{UDELAR}. Este texto ofrece una guía para la presentación de tesis de maestría y doctorado\footnote{A continuación se presenta una caracterización de estos trabajos, de acuerdo con lo estipulado en los artículos correspondientes de la Ordenanza de las Carreras de Posgrado$^1$, aprobada por el CDC en setiembre de 2001: Art 17 - Las carreras de maestría tienen por objetivo proporcionar una formación superior a 	la del graduado universitario, en un campo del conocimiento. Dicho objetivo se logrará 	profundizando la formación teórica, el conocimiento actualizado y especializado en ese 	campo, y de sus métodos; estimulando el aprendizaje autónomo y la iniciativa personal, e incluyendo la preparación de una tesis o trabajo creativo finales. 

\begin{minipage}{0.973\textwidth}
Art. 23 - Por Tesis, se entenderá un trabajo que demuestre por parte del aspirante, haber 	alcanzado el estado actual del conocimiento y competencia conceptual y metodológica.
	
	Art. 26 - Las carreras de doctorado constituyen el nivel superior de formación de posgrado 	en un área del conocimiento. Su objetivo es asegurar la capacidad de acompañar la 	evolución del área de conocimiento correspondiente, una formación amplia y profunda en el 	área elegida, y la capacidad probada para desarrollar investigación original propia y creación 	de nuevo conocimiento.
\end{minipage}	 
}. Provee elementos para unificar cuestiones de estructura y formato del género tesis. Esta Guía tiene dos materiales complementarios que proporcionan modelos informáticos del procesamiento textual para cada una de las partes de la tesis.


La información  recogida en esta Guía surge de los talleres de escritura así como también de material bibliográfico\footnote{Dentro del material bilbiográfico se referencian aquí unos pocos a modo de ejemplo, estando los demás incluidos en la guía, como ser: \cite{guia1}, \cite{guia2} y \cite{guia9}.} específico en escritura académica, que reinterpretamos de manera amplia con el fin de tener en cuenta las distintas tradiciones y servicios de la \gls{UDELAR}. 

Esta guía se estructura de la siguiente manera: 


\begin{itemize}
\item	\underline{elementos pretextuales:} aquellos que anteceden al cuerpo del texto en la tesis.
\item	\underline{elementos textuales:} cuerpo del texto en el que se expone el tema investigado. 
\item	\underline{elementos postextuales:} aquellos que están a continuación del cuerpo del texto.
\end{itemize}


 % Se carga el capítulo 01
  \chapter{Fundamentos teóricos}

Este capítulo incluye la revisión de la literatura, de los enfoques, teorías o conceptos pertinentes en que se fundamenta la investigación. Se basa fundamentalmente en la exposición de otros trabajos sobre el tema estudiado.

En diferentes tradiciones académicas este capítulo recibe distintas denominaciones: Marco teórico, Estado de la cuestión, Estado del arte. El objetivo de este capítulo es guiar al lector en la interpretación de trabajos que se han ocupado previamente de la cuestión central de la tesis u ofrecen herramientas analíticas o interpretativas. 

Algunas disciplinas incluyen aquí objetivos, hipótesis y justificación de la metodología, en tanto que otras exigen capítulos independientes para estos contenidos. Asimismo, algunos trabajos requieren un capítulo propio para los antecedentes de la investigación.



\section{XXXX}

Es usual que en \textit{Fundamentos teóricos} o en otras partes de la tesis el autor incluya vocabulario específico de la disciplina en un glosario.

Un glosario incluye una lista de términos y su explicación sucinta. El objetivo de este apartado es permitirle a un lector especializado en el área, aunque no necesariamente en la temática, comprender con mayor facilidad ciertos términos. 
Se organiza en forma alfabética y en el cuerpo de la obra se lo puede señalar con \textsc{versalita} la primera vez que se menciona, si este tipo de letra no fue utilizado con otro fin.

Se mencionan a modo de ejemplo tres posibles palabras a definir:
%
\begin{itemize}
\item \gls{adjetivo}
\item \gls{adjetivo_re}
\item \gls{adjetivo_cu}
\end{itemize}
 % Se carga el capítulo 02
  \chapter{Metodología}

El objetivo de este capítulo es justificar el diseño metodológico elegido. La finalidad de una metodología bien descrita es explicitar los pasos mediante los cuales se obtienen los resultados, y por tanto el cumplimiento (o no) de los objetivos establecidos, de manera tal que pueda ser replicado por otro investigador. Si corresponde, también se evaluarán problemas metodológicos y se realizarán consideraciones éticas.
En algunas disciplinas este capítulo se denomina Materiales y métodos. 
En caso de que la investigación sea de carácter experimental, se debe especificar la siguiente información:


\begin{itemize}
\item	el tipo de investigación realizada (experimental, descriptiva, estudio de caso, encuesta de opinión, etc.)
\item	el modo de recolección de datos (análisis documental, observación participante o no, entrevista o cuestionario, etc.)
\item	población o sujetos experimentales.
\item	protocolo de investigación, si corresponde  
\end{itemize}

En algunas disciplinas formales no es pertinente la inclusión de un capítulo que recoja una cierta metodología de trabajo. En tales casos, se espera que el tesista haga mención de cuestiones carácter metodológico en la Introducción.

El procesamiento del trabajo metodológico que no es imprescindible para la comprensión del texto puede incluirse en apéndices o anexos (Anexo \ref{Ane1}).



 % Se carga el capítulo 03
  \chapter{Presentación de los datos, Análisis, Discusión}

En algunas disciplinas, el capítulo Presentación de datos va acompañado del análisis o de la discusión de la información (\textit{Presentación y análisis de los datos}; \textit{Resultados y discusión}), en tanto que en otras, \textit{Presentación}, \textit{Análisis} y \textit{Discusión} son capítulos separados.
El objetivo de esta(s) parte(s) de la tesis es presentar los datos recabados y el análisis realizado a la luz de la bibliografía ya revisada. Se puede incluir la interpretación de los resultados (\textit{Discusión}) a partir del análisis de los datos, o también relacionarlos con estudios relevantes que se entienden pertinentes, aun si estos no se han consignado en los \textit{Fundamentos teóricos}, ya que se entiende que al analizar los datos pueden aparecer algunos que no se enmarcan teóricamente o que no se explican en el encuadre teórico o en estudios ya existentes.

Ahora a modo de ejemplo mencionamos el símbolo de los números reales utilizando el comando \verb|\gls{}| \gls{Real} y el comando \verb|\glssymbol{}| \glssymbol{Real}. Otro ejemplo es mencionar el tensor simétrico de tensiones \gls{sigma}, o un valor escalar  \gls{alph} o un conjunto vacío \gls{emptyset}.

También se encuentra la posibilidad de realizar un control de cambios de la tesis, de manera muy simple. En este caso se soporta hasta tres correctores distintos, cuyas iniciales se declaran al inicio con un color distinto asignado. En este caso se tiene los correctores \textcolor{redtex}{JLM}, \textcolor{bluetex}{JH} y \textcolor{greentex}{MC}. Los correctores deberían de recibir el archivo \verb|tesis.tex|, con todos los archivos dependientes de este para poder hacer los cambios y compilar la tesis. 

Este es un ejemplo en donde se puede sugerir agregar texto \added[id=JLM,remark={a este}]{documento}. Otra \added[id=JLM,remark={Acá hay un ejemplo de un texto bastante largo que se agrega como cambio}]{manera} de utilizar el control de cambios sería la de \deleted[id=JH]{agregar} eliminar texto. También se podría utilizar la funcionalidad de agregar notas referidas a cierta oración o temática. \note[id=JH]{Muy buen análisis!}. Por último se encuentra la opción de reemplazar cierto texto por otro \replaced[id=MC]{poner esta oración.}{Mejor quitemos esto.}

\section{Título de sección}

Ejemplo de tabla

\begin{table}[h!]
\centering
\caption{Leyenda de tabla.}
\label{tab:comp}
\begin{tabular}{|c|c|c|}
  \hline
  $t$ (seg) & $x$(t) & $y$(t)\\
  \hline
  1 & 0.0000 & 0.0001\\
  2 & 0.5000 & 0.2498\\
  3 & 1.0000 & 1.0000\\
  4 & 1.5000 & 2.2403\\
  5 & 2.0000 & 4.0010\\
  6 & 2.5000 & 6.2459\\
  \hline
\end{tabular}
\end{table}

Ejemplo de figura.

\begin{figure}[h!]
\label{fig:comp}
\includegraphics[width=.8\textwidth]{imagenes/chap4/x_vs_y}
\caption{Leyenda de figura.}
\end{figure}
Ejemplo de ecuación:
\begin{equation}
y(x)=x^2
\end{equation}
 % Se carga el capítulo 04
  \chapter{Consideraciones finales}

En este capítulo se sintetizan las posturas expuestas en el capítulo anterior. Se retoma la pregunta de investigación y se expresa si los resultados apoyan o no la hipótesis planteada. 

Además, se pueden hacer contribuciones teóricas o metodológicas a la disciplina y recomendaciones para trabajos futuros o para profundizar en el campo, plantear nuevas interrogantes o proponer explicaciones \textit{post hoc}. En algunos trabajos este capítulo se subdivide en otras secciones que presentan algunos de los contenidos mencionados. 
En algunas tradiciones académicas este capítulo recibe distintas denominaciones: \textit{Conclusiones}, \textit{Conclusiones y trabajos a futuro}, \textit{Consideraciones finales y recomendaciones}. 

 % Se carga el capítulo 05
  %Seguir copiando la linea de arriba para agregar más capítulos.



  \backmatter % Comando que generalos apéndices, anexos y bibliografía. NO COMENTAR
  
  \bibliography{bibliografia}
  \bibend
  
  \glosario 		         % Glosario
 
  
  \apenarabicnumbering
  \apenmatter
  \input{apendice/apendice_A}
  \chapter{Imágenes remasterizadas}\label{Ape2}

  \chapter{Entrevistas desgrabadas}\label{Ape3}

  
  \anexarabicnumbering
  \anexmatter
  \chapter{Material legislativo}\label{Ane1}

XXXXX
    
  
  
\end{document}

% ===== FIN DEL DOCUMENTO =====